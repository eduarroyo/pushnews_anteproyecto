\section{Introducción}

% Definición de comunicación corporativa, comunicación corporativa en internet.
Se define \textbf{Comunicación Corporativa} como ``la totalidad de los recursos de comunicación de los que dispone una organización para llegar efectivamente a sus públicos'' \cite{riel_2001}. Este tipo de comunicaciones ha abrazado Internet como uno de sus canales principales y, hoy en día, tanto empresas como instituciones ejercen su presencia a través de webs corporativas, blogs y más recientemente en redes sociales como Twitter, Facebook, Instagram, etc.

% Comparar con las redes sociales
%   · Desventajas de las RRSS: no hay control sobre los cambios en la plataforma, cesión de datos.
Las redes sociales, en particular, parecen uno de los medios preferido por las empresas \cite{linkedin} y administraciones \cite{grande2015casos} como canal de comunicación debido a la gran acogida que tienen entre la población y a que permiten realizar métricas de impacto de las comunicaciones. Sin embargo, presentan algunos inconvenientes que pueden resultar problemáticos para las organizaciones, como por ejemplo:
\begin{itemize}
    \item Obligación a someterse a las condiciones de la plataforma.
    \item Cesión de datos.
    \item Dificultad de la gestión simultánea de múltiples redes.
    \item Imposibles de extender para incluir funciones personalizadas.
\end {itemize}

% Incremento del acceso a internet desde dispositivos móviles.
No se puede hablar sobre estrategias de comunicación corporativa sin mencionar la importancia de las aplicaciones para dispositivos móviles. A la vista del uso que de ellas se realiza para el acceso a la información y el impacto de las aplicaciones \cite{playstore} \cite{appstore}, es conveniente tenerlas en consideración dentro de la estrategia de comunicaciones de cualquier empresa o institución. Las aplicaciones corporativas permiten, además de acceso a los comunicados, ofrecer otros servicios a los usuarios, específicos de la organización. También permiten utilizar notificaciones (mensajes ``push'') son una fuerte llamada de atención para el usuario, lo que supone cierto grado de garantía de éxito de la comunicación.

% Justificación
Sería interesante para las organizaciones contar con una plataforma que les permitiera gestionar de manera centralizada sus comunicaciones, junto con una aplicación móvil personalizada que los usuarios utilizarían para la consulta. Esta plataforma les permitiría cubrir las necesidades comentadas anteriormente, salvando algunos de los inconvenientes de las redes sociales al tiempo que se mantiene abierta para la integración de servicios propios sin perder de vista la interconexión con redes sociales.

Teniendo en cuenta lo anteriormente expuesto, se propone como tema de Trabajo de Fin de Grado la realización de una plataforma web de gestión de comunicados para empresas e instituciones, así como una aplicación tipo de consulta para los usuarios.