\section{Fases del desarrollo}
El desarrollo del proyecto comprende las siguientes fases diferenciadas:

\subsection{Estudio}
Durante esta etapa se revisarán y completarán los conocimientos sobre las tecnologías fundamentales que van a ser empleadas en el proyecto, que se pueden clasificar como tecnologías del lado del servidor: ASP.NET MVC5, Entity Framework, SQL Server y tecnologías del lado del cliente: HTML5, CSS3, Javascript y los frameworks a emplear, así como la plataforma de desarrollo para la aplicación tipo.

\subsection{Análisis}
El objetivo de esta fase es conocer la naturaleza del producto que se va a construir. Para ello se han de determinar los requisitos funcionales y no funcionales del sistema, es decir: qué debe hacer el sistema. Se podrán utilizar métodos diversos con el fin de mejorar la efectividad  del proceso y garantizar la veracidad de los requisitos.

Para la obtención de los requisitos funcionales se elaborará un modelo de casos de uso que permitirá identificar los actores y escenarios que forman parte del sistema.

Para la obtención de los requisitos no funcionales podrán elaborarse prototipos de los elementos del sistema que representen mayor riesgo, con el fin de poder evaluar parámetros como el desempeño y formar una idea del dimensionamiento del sistema.

\subsection{Diseño}
Durante la fase de diseño se elaborará un modelo formal del sistema atendiendo a los requisitos obtenidos durante la fase anterior. Este modelo deberá ser apto para ser traducido a código en la etapa siguiente.

Se llevará a cabo en dos etapas: una preliminar que dará lugar a la arquitectura del sistema y a una aproximación al diseño de datos e interfaz de usuario; y otra más detallada que se centrará en los procedimientos y donde se refinará el diseño de datos e interfaz.

\subsection{Codificación}
La fase de codificación consistirá en la construcción de la solución, que deberá ser fiel al diseño elaborado en la fase anterior. La solución tendrá la forma de aplicación WEB basada en tecnologías .NET. Se utilizará C\string# como lenguaje del lado del servidor y HTML, CSS, y javascript del lado del cliente.

Por su parte, la aplicación tipo se construirá empleando un framework de código compartido que permite crear aplicaciones nativas para múltiples plataformas a partir de una única base de código, que consumirá los datos a través de un servicio web basado en .NET.

\subsection{Prueba}
La fase de prueba tiene como objetivo verificar que la solución construida cumple los requisitos establecidos en la fase de análisis.

\subsection{Documentación}
En la fase de documentación se elaborará la Memoria Final de Proyecto, así como los manuales de usuario y de código.

\subsection{Distribución temporal}
El Trabajo de Fin de Grado supone 12 créditos según el plan de estudios vigente. Considerando 25 horas por crédito, sumarían un total de 300 horas para su realización. La distribución temporal aproximada de las horas entre las diferentes fases del trabajo se detalla en la siguiente tabla:

\begin{table}[h!]
    \centering
    \begin{tabular}{r|rrrrrr|r}
                       & Estudio & Análisis & Diseño & Codificación & Prueba & Documentación & \textbf{Total} \\
        \hline         
        marzo          &      12 &       25 &     32 &              &        &            37 &           106 \\
        abril          &         &          &     30 &           55 &     15 &            15 &           115 \\
        mayo           &         &          &        &           30 &     23 &            26 &            79 \\
        \hline
        \textbf{Total} &      12 &       25 &     62 &           85 &     38 &            78 &           300
    \end{tabular}
    \caption{Distribución de horas en meses y tareas}
\end{table}